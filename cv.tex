\documentclass{article}

\usepackage[french,english]{babel}
\usepackage{eurosym}
\usepackage{iflang}
\usepackage{resume}
\usepackage{hyperref}
\begin{document}
\selectlanguage{english}

\name{\bf Corentin LE MOLGAT}

T\'{e}l: +33 6 66 89 42 07\\
\href{mailto:corentin.lemolgat@gmail.com}{\texttt{corentin.lemolgat@gmail.com}}\\
\IfLanguageName{english} {
Japanese spouse visa holder\\
} {
}

\begin{llist}
\sectiontitle{Education}
\employer{ENSEIRB} \location{Bordeaux, France}
\dates{09/2006--10/2009}
\IfLanguageName{english} {
B.S./M.S. in Computer Science with a specialty in HPC (High Performance Computing).
} {
Dipl\^{o}me d'ing\'{e}nieur en Informatique option PRCD (Parall\'{e}lisme,
R\'{e}gulation et Calcul Distribu\'{e}).
}

% Working Experience
\IfLanguageName{english} {
\sectiontitle{Work Experience}
} {
\sectiontitle{Exp\'{e}rience Professionnelle}
}
\vspace{-0.33cm}

\IfLanguageName{english} {
\employer{Kelly Services as contractor at Google LLC}
} {
\employer{Kelly Services, CDD \`{a} Google LLC}
}
\location{Paris, France}
\dates{11/2017--05/2019}
\IfLanguageName{english} {
Open Source Software Release Manager \\
\vspace{-0.33cm}

Optimization Team:
\vspace{-0.33cm}
\begin{itemize}
	\item Released several versions of \href{https://github.com/google/or-tools}{\texttt{Google OR-Tools}} (PyPI, Nuget, GitHub).
	\item Reworked and maintained the online documentation (HTML, markdown, doxygen).
	\item Provided support to customer (GitHub issues, Stack Overflow).
	\item Developed and maintained few samples (C++, Java, Python, .Net).
	\item Maintained the Makefile based build system (Makefile, bash).
	\item Developed and maintained the CMake based build system (CMake).
	\item Developed and maintained a public CI system (Travis CI, Appveyor, Docker).
	\item Provided training \& support for CMake.
\end{itemize}
} {
Open Source Software Release Manager \\
\vspace{-0.33cm}

Equipe Optimization:
\vspace{-0.33cm}
\begin{itemize}
	\item Publication de plusieurs versions de
		\href{https://github.com/google/or-tools}{\texttt{Google OR-Tools}} (PyPI, Nuget, GitHub).
  \item Refonte et maintenance de la documention en ligne (HTML, markdown, doxygen).
  \item Support proposé aux utilisateurs (GitHub issues, Stack Overflow).
  \item D\'{e}veloppement et maintenance de plusieurs examples (C++, Java, Python, .Net).
  \item Maintenance du syst\`{e}me de build utilisant Makefile (Makefile, bash).
  \item D\'{e}veloppement et maintenance du syst\`{e}me de build utilisant CMake (CMake).
  \item D\'{e}veloppement et maintenance du syst\`{e}me d'int\'{e}gration continue publique (Travis CI, Appveyor, Docker).
  \item Formation \& support apportés sur CMake.
\end{itemize}
}

\employer{Aldebaran Robotics/SoftBank Robotics Europe}\location{Paris, France}
\dates{04/2012--11/2017}
\IfLanguageName{english} {
Embedded System \& Computer Vision Software Engineer \\
\vspace{-0.33cm}

System Team:
\vspace{-0.33cm}
\begin{itemize}
	\item Developed and maintained a Kernel Linux SoC Driver (C, MT9M114, OV5640).
	\item Developed a camera firmware flasher (Archlinux \& Yocto, C++).
	\item Managed a contractor for an UVC compliant firmware (CMake, Docker, C++, Catch, GTest, Plantuml).
\end{itemize}
Vision Team:
\vspace{-0.33cm}
\begin{itemize}
	\item Reworked and maintained a C++ framework for multi-client access to robot cameras (CMake, C++, Boost).
	\item Developed tooling for a camera viewer (C++, Qt).
	\item Developed Modularity, a C++ computational graph framework for perception.
	\item Maintained the internal CI builfarm, testing and training (Jenkins,
		gcovr).
	\item Provided training \& support for CMake and C++ as a senior developer.
\end{itemize}
Misc:
\vspace{-0.33cm}
\begin{itemize}
	\item Supported the vision system for the R \& D team and research partners.
	\item Provided support on the production line, Yantai (China), 1 month.
\end{itemize}
} {
R\&D Ing\'{e}nieur Informaticien Vision et Syst\`{e}me embarqu\'{e}
\vspace{-0.33cm}

Equipe Syst\`{e}me:
\vspace{-0.33cm}
\begin{itemize}
	\item D\'{e}veloppement et maintenance d'un driver SoC pour un noyau Linux (C, MT9M114, OV5640).
	\item D\'{e}veloppement d'un chargeur de micrologiciel (firmware) pour une cam\'{e}ra (Archlinux \& Yocto, C++).
	\item Gestion d'un fournisseur pour un micrologiciel compatible UVC (CMake, Docker, C++, Catch, GTest, Plantuml).
\end{itemize}
Equipe Vision:
\vspace{-0.33cm}
\begin{itemize}
	\item Refonte et maintenance d'un framework C++ pour l'acc\`{e}s multi-clients aux cam\'{e}ras du robot (CMake, C++, Boost).
	\item D\'{e}velopement d'outils de visualisation pour cameras (C++, Qt).
	\item D\'{e}velopement de Modularity, un framework C++ de graph de calcul pour Perception.
	\item Maintenance d'une builfarm d'int\`{e}gration continue interne, test et formation (Jenkins, gcovr).
	\item Formation \& support sur CMake et C++ en tant que d\'{e}veloppeur s\'{e}nior.
\end{itemize}
Divers:
\vspace{-0.33cm}
\begin{itemize}
	\item Support et formation sur l'utilisation du syst\`{e}me de vision pour l'\'{e}quipe de R \& D et les partenaires de recherche.
	\item Support sur la ligne de production, Yantai (Chine), 1 mois.
\end{itemize}
}

\employer{Vi Technology}\location{St-Egr\`{e}ve, France}
\dates{02/2010--12/2011}
\IfLanguageName{english} {
R\&D GPGPU and Vision System Software Engineer\\
\vspace{-0.33cm}

Responsible for the design and development of the whole acquisition and processing pipeline
for a new AOI (Automated Optical Inspection) system for SPI (Solder Past Inspection)
 running on Linux (Fedora).

% Hardware
Software lead for the hardware acquisition system integration (Vertex-6 Card on PCIe):
\vspace{-0.33cm}
\begin{itemize}
	\item Managed the integration of the FPGA.
	\item Defined the protocol between the Kernel and the acquisition card.
	\item Developed the Kernel device driver (C).
	\item Developed debugger tools (C++, Qt).
\end{itemize}
% Software
Software lead on the image pipeline:
\vspace{-0.33cm}
\begin{itemize}
	\item Developed a C++ middleware to grab and manage images from several dozens of image sensors.
	\item Managed two co-workers to speed up development (roadmap, code review, scrum master).
	\item Developed a 2D camera image viewer (C++, Qt, OpenSceneGraph).
\end{itemize}
% GPGPU
Software lead on the GPGPU post-processing pipeline:
\vspace{-0.33cm}
\begin{itemize}
	\item Ported the 3D reconstruction algorithm (Matlab) to a dual-GPU System (CMake, C++, CUDA 4, GTX 480)
		and speed it up from 15s to 7ms (x2000!).
	\item Developed a CMake cross toolchain for managing CUDA files.
	\item Developed a 3D PCB viewer (after 3D reconstruction) using (C++, Qt, OpenSceneGraph).
\end{itemize}

Various support as technical lead on GNU/Linux:
\vspace{-0.33cm}
\begin{itemize}
	\item CMake training \& support.
	\item Jenkins training \& support (POC, setup, design).
	\item Linux training \& support (Bash, Fedora) (everyone else was on Windows).
\end{itemize}
} {
R\&D GPGPU and Vision System Software Engineer\\
\vspace{-0.33cm}

Responsable de l'architecture et du d\'{e}veloppement de l'ensemble de la chaine d'acquisition et de traitement
pour une nouvelle machine d'AOI (Automated Optical Inspection) pour l'\'{e}tape de SPI (Solder Past Inspection)
tournant sous GNU/Linux (Fedora).

% Hardware
Software lead pour l'int\'{e}gration du syst\`{e}me d'acquisition mat\'{e}riel (Vertex-6 Card on PCIe):
\vspace{-0.33cm}
\begin{itemize}
	\item Gestion de l'int\'{e}gration du FPGA.
	\item D\'{e}finition du protocole entre le noyau Linux et la carte d'acquisition.
	\item D\'{e}veloppement du pilote de p\'{e}riph\'{e}rique du noyau (C).
	\item D\'{e}veloppement d'un outils de d\'{e}bogage (C++, Qt).
\end{itemize}
% Software
Software lead sur le pipeline d'image:
\vspace{-0.33cm}
\begin{itemize}
	\item D\'{e}veloppement d'un C++ middleware to grab and manage images from several dozens of image sensors.
	\item Gestion de deux co-workers pour acc\'{e}l\'{e}rer le d\'{e}veloppement (roadmap, code review, scrum master).
	\item D\'{e}veloppement d'un visualiseur d'image de cam\'{e}ra 2D (C++, Qt, OpenSceneGraph).
\end{itemize}
% GPGPU
Software lead sur le pipeline de post-processing GPGPU:
\vspace{-0.33cm}
\begin{itemize}
	\item Portage de l'agorithme de reconstruction 3D (Matlab) vers un syst\`{e}me bi-GPU (CMake, C++, CUDA 4, GTX 480)
		et acc\'{e}l\'{e}ration de 15s \`{a} 7ms (x2000!).
	\item D\'{e}veloppement d'une cross toolchain CMake pour g\'{e}rer les fichiers CUDA.
	\item D\'{e}veloppement d'un visualiseur 3D de PCB apr\'{e}s reconstruction 3D (C++, Qt, OpenSceneGraph).
\end{itemize}

Supports divers commme responsable technique sur GNU/Linux:
\vspace{-0.33cm}
\begin{itemize}
	\item CMake formation \& support.
	\item Jenkins formation \& support (POC, setup, design).
	\item Linux formation \& support (Bash, Fedora) (tout le monde \'{e}taient sur Windows).
\end{itemize}
}

% Internship
\employer{Kyushu University}\location{Fukuoka, Japan}
\dates{04/2009--10/2009}
\IfLanguageName{english} {
Engineering intern at I.R.V.S. (laboratory for Intelligent Robots \& Vision System)
\vspace{-0.33cm}
\begin{itemize}
	\item Design (UML), implementation (C++) and tooling viewer (C++, Qt) of a
		3D human pose estimation using non-parametric belief propagation
		algorithms and multiple 2D video cameras.
\end{itemize}
} {
Stage d'ing\'{e}nieur \`{a} l'I.R.V.S. (laboratory For Intelligent Robots \& Vision System)
\vspace{-0.33cm}
\begin{itemize}
	\item Design (UML), implementation (C++) and tooling viewer (C++, Qt) of a
		3D human pose estimation using non-parametric belief propagation
		algorithms and multiple 2D video cameras.
\end{itemize}
}

\employer{Kyushu University}\location{Fukuoka, Japan}
\dates{06/2008--09/2008}
\IfLanguageName{english} {
Engineering intern at I.R.V.S. (laboratory For Intelligent Robots \& Vision System)
\vspace{-0.33cm}
\begin{itemize}
	\item 3D Reconstruction on GPU (GLSL) and tooling viewer (C++, Qt) using stereovision algorithms and four 2D video cameras.
\end{itemize}
} {
Stage d'ing\'{e}nieur \`{a} l'I.R.V.S. (laboratory For Intelligent Robots \& Vision System)
\vspace{-0.33cm}
\begin{itemize}
	\item 3D Reconstruction on GPU (GLSL) and tooling viewer (C++, Qt) using stereovision algorithms and four 2D video cameras.
\end{itemize}
}

% Skills
\IfLanguageName{english} {
\sectiontitle{Skills}
{\em Competence}: Programming, CI, Documentation, Tooling, Image Pipeline, Drivers, Robotics \\
{\em Programming Languages}: Bash, Kernel C, C++, Python, Java, .Net, Docker \\
{\em Programming Libraries}: STL, OpenCV, Boost, V4L2, ffmpeg \\
{\em Extra Interests}: Android, Vulkan, Spir V \\
{\em Languages}: French (native), English (fluent), Japanese (beginner), Spanish (beginner) \\
} {
\sectiontitle{Comp\'{e}tences}
{\em Comp\'{e}tences}: Programmation, CI, Outillage, Drivers, Vision, Robotique \\
{\em Langages de Programmation}: Kernel C, C++, Bash, Python, Java, .Net, Docker \\
{\em Biblioth\`{e}ques Logicielles}: STL, OpenCV, Boost, V4L2, ffmpeg \\
{\em Autres Int\'{e}r\^{e}ts}: Android, Vulkan, Spir V \\
{\em Langues}: Fran\c{c}ais (natif), Anglais (fluent), Espagnol (d\'{e}butant), Japonais (d\'{e}butant) \\
}

% References
\IfLanguageName{english} {
\sectiontitle{References}
} {
\sectiontitle{R\'{e}f\'{e}rences}
}
\textbf{Dr. RABAUD Vincent}\\
Google, Inc. \\
vrabaud@google.com

\textbf{Dr. PERRON Laurent}\\
Google, Inc. \\
lperron@google.com

\end{llist}

{\em Last update: \today}

\end{document}
