\documentclass{article}

\usepackage[french,english]{babel}
\usepackage{eurosym}
\usepackage{iflang}
\usepackage{resume}
\usepackage{hyperref}
\begin{document}
\selectlanguage{english}

\name{\bf Corentin LE MOLGAT}

T\'{e}l: +33 6 66 89 42 07\\
\href{mailto:corentin.lemolgat@gmail.com}{\texttt{corentin.lemolgat@gmail.com}}\\
Japanese spouse visa holder\\

\begin{llist}
\sectiontitle{Education}
\employer{ENSEIRB} \location{Bordeaux, France}
\dates{09/2006--10/2009}
\IfLanguageName{english} {
B.S./M.S. in Computer Science with a specialty in HPC (High Performance Computing).
} {
Dipl\^{o}me d'ing\'{e}nieur en Informatique option PRCD (Parall\'{e}lisme,
R\'{e}gulation et Calcul Distribu\'{e}).
}

% Working Experience
\IfLanguageName{english}
{\sectiontitle{Work Experience}}
{\sectiontitle{Exp\'{e}rience Professionnelle}}
\vspace{-0.33cm}

\employer{Aldebaran Robotics/SoftBank Robotics Europe}\location{Paris, France}
\dates{04/2012--now}
\IfLanguageName{english} {
R\&D Computer Vision \& Embedded System Software Engineer\\
\vspace{-0.33cm}

System Team:
\begin{itemize}
\item developed and maintained a Kernel Linux SoC Driver (C, MT9M114, OV5640).
\item developed a camera firmware flasher (Gentoo \& Yocto, C++).
\item managed a contractor for an UVC compliant firmware (CMake, Docker, C++, Catch, Plantuml).
\end{itemize}
Vision Team:
\begin{itemize}
\item reworked and maintained a C++ framework for multi-client access to robot cameras (CMake, C++, Boost).
\item developed tooling for a camera viewer (C++, Qt).
\item developed Modularity, a C++ computational graph framework for perception.
\item maintained the internal CI builfarm, testing and training (Jenkins,
 gcovr).
\item provided training \& support for CMake as a senior developer.
\item provided training \& support for C++ as a senior developer.
\end{itemize}
Misc:
\begin{itemize}
\item supported the vision system for the R \& D team and research partners.
\item provided support on the production line, Yantai (China), 1 month.
\end{itemize}
}{
R\&D Ing\'{e}nieur Informaticien Vision et Syst\`{e}me embarqu\'{e}
\vspace{-0.33cm}
Equipe Syst\`{e}me:

Equipe Vision:

}

\employer{Vi Technology}\location{St-Egr\`{e}ve, France}
\dates{02/2010--12/2011}
\IfLanguageName{english} {
R\&D GPGPU and Vision System Software Engineer\\
\vspace{-0.33cm}

Responsible for the design and development of the whole acquisition and processing pipeline
for a new AOI (Automated Optical Inspection) system for SPI (Solder Past Inspection)
 running on Linux (Fedora).
% Hardware
\item Software lead for the hardware acquisition system integration (Vertex-6 Card on PCIe):
\begin{itemize}
\item managed the integration of the FPGA.
\item defined the protocol between the Kernel and the acquisition card.
\item developed the Kernel device driver (C).
\item developed debugger tools (C++, Qt).
\end{itemize}
% Software
\item Software lead on the image pipeline:
\begin{itemize}
\item developed a C++ middleware to grab and manage images from several dozens of image sensors.
\item managed two co-workers to speed up development (roadmap, code review, scrum master)
\item developed a 3D PCB viewer (after 3D reconstruction) using (C++, Qt, OpenSceneGraph).
\item developed a 2D camera image viewer (C++, Qt, OpenSceneGraph).
\end{itemize}
% GPGPU
\item Software lead on the GPGPU post-processing pipeline:
\begin{itemize}
\item ported the 3D reconstruction algorithm (Matlab) to a dual-GPU System (CMake, C++, CUDA 4, GTX 480)
 and sped it up from 15s to 7ms (x2000!).
\item developed a CMake cross toolchain for managing CUDA files.
\end{itemize}

Various support as technical lead on GNU/Linux:
\begin{itemize}
\item CMake training \& support.
\item Jenkins training \& support (POC, setup, design).
\item Linux training \& support (Bash, Fedora) (everyone else was on Windows).
\end{itemize}
} {}

% Internship
\employer{Kyushu University}\location{Fukuoka, Japan}
\dates{04/2009--10/2009}
\IfLanguageName{english} {
Engineering intern at I.R.V.S. (laboratory for Intelligent Robots \& Vision System)
\vspace{-0.33cm}
\begin{itemize}
\item design (UML), implementation (C++) and tooling viewer (C++, Qt) of a
3D human pose estimation using non-parametric belief propagation
algorithms and multiple 2D video cameras.
\end{itemize}
} {}

\employer{Kyushu University}\location{Fukuoka, Japan}
\dates{06/2008--09/2008}
\IfLanguageName{english} {
Engineering intern at I.R.V.S. (laboratory For Intelligent Robots \& Vision System)
\vspace{-0.33cm}
\begin{itemize}
\item 3D Reconstruction on GPU (GLSL) and tooling viewer (C++, Qt) using stereovision algorithms and four 2D video cameras.
\end{itemize}
} {}

% Skills
\IfLanguageName{english} {
\sectiontitle{Skills}
{\em Competence}: Image Pipeline, Drivers, Robotics, Programming, Architecture, Management \\
{\em Programming Languages}: Kernel C, C++ \\
{\em Programming Libraries}: STL, OpenCV, Boost, V4L2, ffmpeg \\
{\em Extra Interests}: Android, CUDA, GLSL \\
{\em Languages}: French (native), English (fluent), Japanese (beginner), Spanish
(beginner) \\
} {
\sectiontitle{Comp\'{e}tences}
{\em Comp\'{e}tences}: Drivers, Vision, Robotique, Programmation, Architecte Logiciel,
Management \\
{\em Langages de Programmation}: Kernel C, C++ \\
{\em Biblioth\`{e}ques Logicielles}: STL, OpenCV, Boost, V4L2, ffmpeg \\
{\em Autres Int\'{e}r\^{e}ts}: Android, CUDA, GLSL \\
{\em Langues}: Fran\c{c}ais (natif), Anglais (fluent), Espagnol (d\'{e}butant),
Japonais (d\'{e}butant) \\
}

% References
\sectiontitle{References}
\textbf{Dr. RABAUD Vincent}\\
Google, Inc. \\
vrabaud@google.com

\textbf{Dr. PERRON Laurent}\\
Google, Inc. \\
lperron@google.com

\end{llist}

{\em Last update: \today}

\end{document}
