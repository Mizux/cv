%%%%%%%%%%%%%%%%%%%%%%%%%%%%%%%%%%%%%%%%%%%%%%%%%%%%%%%%%%
%                Identification
%%%%%%%%%%%%%%%%%%%%%%%%%%%%%%%%%%%%%%%%%%%%%%%%%%%%%%%%%%
\documentclass{resume}
\geometry{a4paper,portrait,margin=1.0cm}
%\geometry{letterpaper,portrait,margin=1.0cm}
\begin{document}
\selectlanguage{english}
%%%%%%%%%%%%%%%%%%%%%%%%%%%%%%%%%%%%%%%%%%%%%%%%%%%%%%%%%%
%                Contact info
%%%%%%%%%%%%%%%%%%%%%%%%%%%%%%%%%%%%%%%%%%%%%%%%%%%%%%%%%%
\ContactName
{Corentin LE MOLGAT} % first and last name

\iffalse
%\iftrue
\ContactInfoPhysical
{51 Rue du Docteur Denoyelle} % house and street
{Tours, Centre, 37000}        % city, state, zip
{France}                      % country
\fi

\ContactInfoDigital
  {tel:+33666894207}                    % phone URL (with int'l code)
  {06-66-89-42-07}                      % phone display text
  {mailto:corentin.lemolgat@gmail.com}  % email URL
  {corentin.lemolgat@gmail.com}         %email email display text
  {https://www.linkedin.com/in/corentin-le-molgat-a957893b/} % LinkedIn URL
  {Corentin Le Molgat}                                       % LinkedIn display text
  {https://github.com/Mizux} % GitHub URL
  {Mizux}                    % GitHub display text

%%%%%%%%%%%%%%%%%%%%%%%%%%%%%%%%%%%%%%%%%%%%%%%%%%%%%%%%%%
%                Work experience
%%%%%%%%%%%%%%%%%%%%%%%%%%%%%%%%%%%%%%%%%%%%%%%%%%%%%%%%%%
\IfLanguageName{english} {
  \section{{\faBriefcase} Work Experience}
} {
  \section{{\faBriefcase} Exp\'{e}rience Professionnelle}
}

\IfLanguageName{english} {
  \PlaceAndLocation{Adecco, Contractor at Google LLC}{Paris, France} \\
  \TitleAndYears{Open Source Software Developer}{01 2020 - 01 2021} \\
  \PlaceAndLocation{Kelly Services, Contractor at Google LLC}{Paris, France} \\
  \TitleAndYears{Open Source Software Release Manager}{11 2017 - 05 2019} \\
  Optimization Team:\\
  \begin{itemize}
  	\item Released several versions of
      \href{https://github.com/google/or-tools}{\texttt{Google OR-Tools}} (PyPI, Nuget, GitHub).
  	\item Reworked and maintained the
      \href{https://developers.google.com/optimization/}{\texttt{online documentation}}
      (HTML, markdown, doxygen, pydoc).
  	\item Developed and maintained the three build systems (Bazel, CMake, Makefile).
  	\item Developed and maintained samples (C++, Java, Python, .Net).
  	\item Developed and maintained the public CI systems
      (GitHub workflow, Travis CI, Appveyor, Docker).
  	\item Provided support to customer (GitHub issues, Stack Overflow, Discord).
  	\item Provided training \& support for CMake.
  \end{itemize}
} {
  \PlaceAndLocation{Adecco, CDD \`{a} Google LLC}{Paris, France} \\
  \TitleAndYears{Open Source Software Developer}{01 2020 - 01 2021} \\
  \PlaceAndLocation{Kelly Services, CDD \`{a} Google LLC}{Paris, France} \\
  \TitleAndYears{Open Source Software Release Manager}{01 2020 - 01 2021} \\
  Equipe Optimization:\\
  \begin{itemize}
  	\item Publication de plusieurs versions de
  		\href{https://github.com/google/or-tools}{\texttt{Google OR-Tools}} (PyPI, Nuget, GitHub).
    \item Refonte et maintenance de la
      \href{https://developers.google.com/optimization/}{\texttt{documention en ligne}}
      (HTML, markdown, doxygen).
    \item D\'{e}veloppement et maintenance des 3 syst\`{e}mes de build (Bazel, CMake, Makefile).
    \item D\'{e}veloppement et maintenance des examples (C++, Java, Python, .Net).
    \item D\'{e}veloppement et maintenance des syst\`{e}mes d'int\'{e}gration continue publiques
      (GitHub workflow, Travis CI, Appveyor, Docker).
    \item Support propos\'{e} aux utilisateurs (GitHub issues, Stack Overflow, Discord).
    \item Formation \& support apport\'{e} sur CMake.
  \end{itemize}
}

\bigskip

\PlaceAndLocation{Aldebaran Robotics/SoftBank Robotics Europe}{Paris, France} \\
\IfLanguageName{english} {
  \TitleAndYears{Embedded System \& Computer Vision Software Engineer}{04 2012 - 11 2017} \\
  System Team:\\
  \begin{itemize}
  	\item Developed and maintained a Kernel Linux SoC Driver (C, MT9M114, OV5640).
  	\item Developed a camera firmware flasher (Archlinux \& Yocto, C++).
  	\item Managed a contractor for an UVC compliant firmware (CMake, Docker, C++, Catch, GTest, Plantuml).
  \end{itemize}
  Vision Team:\\
  \begin{itemize}
  	\item Reworked and maintained a C++ framework for multi-client access to robot cameras (CMake, C++, Boost).
  	\item Developed tooling for a camera viewer (C++, Qt).
  	\item Developed Modularity, a C++ computational graph framework for perception.
  	\item Maintained the internal CI builfarm, testing and training (Jenkins,
  		gcovr).
  	\item Provided training \& support for CMake and C++ as a senior developer.
  \end{itemize}
  Misc:\\
  \begin{itemize}
  	\item Supported the vision system for the R \& D team and research partners.
  	\item Provided support on the production line, Yantai (China), 1 month.
  \end{itemize}
} {
  \TitleAndYears{R\&D Ing\'{e}nieur Informaticien Vision par Ordinateur et Syst\`{e}me}{04 2012 - 11 2017} \\
  Equipe Syst\`{e}me:\\
  \begin{itemize}
  	\item D\'{e}veloppement et maintenance d'un driver SoC pour un noyau Linux (C, MT9M114, OV5640).
  	\item D\'{e}veloppement d'un chargeur de micrologiciel (firmware) pour une cam\'{e}ra (Archlinux \& Yocto, C++).
  	\item Gestion d'un fournisseur pour un micrologiciel compatible UVC (CMake, Docker, C++, Catch, GTest, Plantuml).
  \end{itemize}
  Equipe Vision:\\
  \begin{itemize}
  	\item Refonte et maintenance d'un framework C++ pour l'acc\`{e}s multi-clients aux cam\'{e}ras du robot (CMake, C++, Boost).
  	\item D\'{e}velopement d'outils de visualisation pour cameras (C++, Qt).
  	\item D\'{e}velopement de Modularity, un framework C++ de graph de calcul pour Perception.
  	\item Maintenance d'une builfarm d'int\`{e}gration continue interne, test et formation (Jenkins, gcovr).
  	\item Formation \& support sur CMake et C++ en tant que d\'{e}veloppeur s\'{e}nior.
  \end{itemize}
  Divers:\\
  \begin{itemize}
  	\item Support et formation sur l'utilisation du syst\`{e}me de vision pour l'\'{e}quipe de R \& D et les partenaires de recherche.
  	\item Support sur la ligne de production, Yantai (Chine), 1 mois.
  \end{itemize}
}

\bigskip

\PlaceAndLocation{Vi Technology}{St-Egr\`{e}ve, France} \\
\IfLanguageName{english} {
  \TitleAndYears{R\&D GPGPU and Vision System Software Engineer}{02 2010 - 12 2011} \\
  Responsible for the design and development of the whole acquisition and processing pipeline
  for a new AOI (Automated Optical Inspection) system for SPI (Solder Past Inspection)
   running on Linux (Fedora).\\
  % Hardware
  Software lead for the hardware acquisition system integration (Vertex-6 Card on PCIe):\\
  \begin{itemize}
  	\item Managed the integration of the FPGA.
  	\item Defined the protocol between the Kernel and the acquisition card.
  	\item Developed the Kernel device driver (C).
  	\item Developed debugger tools (C++, Qt).
  \end{itemize}
  % Software
  Software lead on the image pipeline:\\
  \begin{itemize}
  	\item Developed a C++ middleware to grab and manage images from several dozens of image sensors.
  	\item Managed two co-workers to speed up development (roadmap, code review, scrum master).
  	\item Developed a 2D camera image viewer (C++, Qt, OpenSceneGraph).
  \end{itemize}
  % GPGPU
  Software lead on the GPGPU post-processing pipeline:\\
  \begin{itemize}
  	\item Ported the 3D reconstruction algorithm (Matlab) to a dual-GPU System (CMake, C++, CUDA 4, GTX 480)
  		and speed it up from 15s to 7ms (x2000!).
  	\item Developed a CMake cross toolchain for managing CUDA files.
  	\item Developed a 3D PCB viewer (after 3D reconstruction) using (C++, Qt, OpenSceneGraph).
  \end{itemize}
  Various support as technical lead on GNU/Linux:\\
  \begin{itemize}
  	\item CMake training \& support.
  	\item Jenkins training \& support (PoC, setup, design).
  	\item Linux training \& support (Bash, Fedora) (everyone else was on Windows).
  \end{itemize}
} {
  \TitleAndYears{R\&D Ing\'{e}nieur Informaticien Syst\`{e}me de Vision et GPGPU}{02 2010 - 12 2011} \\
  Responsable de l'architecture et du d\'{e}veloppement de l'ensemble de la chaine d'acquisition et de traitement
  pour une nouvelle machine d'AOI (Automated Optical Inspection) pour l'\'{e}tape de SPI (Solder Past Inspection)
  tournant sous GNU/Linux (Fedora).\\
  % Hardware
  Software lead pour l'int\'{e}gration du syst\`{e}me d'acquisition mat\'{e}riel (Vertex-6 Card on PCIe):\\
  \begin{itemize}
  	\item Gestion de l'int\'{e}gration du FPGA.
  	\item D\'{e}finition du protocole entre le noyau Linux et la carte d'acquisition.
  	\item D\'{e}veloppement du pilote de p\'{e}riph\'{e}rique du noyau (C).
  	\item D\'{e}veloppement d'un outils de d\'{e}bogage (C++, Qt).
  \end{itemize}
  % Software
  Software lead sur le pipeline d'image:\\
  \begin{itemize}
  	\item D\'{e}veloppement d'un middleware d'acquisition et de gestion des
  		images depuis deux douzaines de capteurs CMOS (C++).
  	\item Gestion de deux co-workers pour acc\'{e}l\'{e}rer le d\'{e}veloppement (roadmap, code review, scrum master).
  	\item D\'{e}veloppement d'un visualiseur d'image de cam\'{e}ra 2D (C++, Qt, OpenSceneGraph).
  \end{itemize}
  % GPGPU
  Software lead sur le pipeline de post-processing GPGPU:\\
  \begin{itemize}
  	\item Portage de l'agorithme de reconstruction 3D (Matlab) vers un syst\`{e}me bi-GPU (CMake, C++, CUDA 4, GTX 480)
  		et acc\'{e}l\'{e}ration de 15s \`{a} 7ms (x2000!).
  	\item D\'{e}veloppement d'une cross toolchain CMake pour g\'{e}rer les fichiers CUDA.
  	\item D\'{e}veloppement d'un visualiseur 3D de PCB apr\'{e}s reconstruction 3D (C++, Qt, OpenSceneGraph).
  \end{itemize}
  Supports divers commme responsable technique sur GNU/Linux:\\
  \begin{itemize}
  	\item CMake formation \& support.
  	\item Jenkins formation \& support (PoC, setup, design).
  	\item Linux formation \& support (Bash, Fedora) (tout le monde \'{e}taient sur Windows).
  \end{itemize}
}

\bigskip

\IfLanguageName{english} {
  \PlaceAndLocation{Kyushu University}{Fukuoka, Japan} \\
  \TitleAndYears{Software Engineering intern}{04 2009 - 10 2009} \\
  Engineering intern at I.R.V.S. (laboratory for Intelligent Robots \& Vision System). \\
  \begin{itemize}
  	\item Design (UML) and implementation (C++) of a
  		3D human pose estimation using non-parametric belief propagation
  		algorithms and multiple 2D video cameras.
  	\item Developed a tooling viewer (C++, Qt, OpenSceneGraph).
  \end{itemize}
} {
  \PlaceAndLocation{Universit\'e de Kyushu}{Fukuoka, Japon} \\
  \TitleAndYears{Stage d'ing\'{e}nieur informaticien}{04 2009 - 10 2009} \\
  Stage d'ing\'{e}nieur \`{a} l'I.R.V.S. (laboratory For Intelligent Robots \& Vision System) \\
  \begin{itemize}
  	\item Conception (UML) et impl\'{e}mentation (C++) d'un algorithme
  		d'estimation de pose humaine en 3D par propagation de croyance
  		non-param\'{e}trique \`{a} partir plusieurs flux vid\'{e}o de cam\'{e}ras 2D.
  	\item D\'{e}veloppement d'une application de visualisation (C++, Qt, OpenSceneGraph).
  \end{itemize}
}

\bigskip

\IfLanguageName{english} {
  \PlaceAndLocation{Kyushu University}{Fukuoka, Japan} \\
  \TitleAndYears{Software Engineering intern}{06 2008 - 09 2008} \\
  Engineering intern at I.R.V.S. (laboratory for Intelligent Robots \& Vision System). \\
  \begin{itemize}
  	\item 3D Reconstruction on GPU (GLSL) using stereovision algorithms and four 2D video cameras.
  	\item Developed a tooling viewer (C++, Qt).
  \end{itemize}
} {
  \PlaceAndLocation{Universit\'e de Kyushu}{Fukuoka, Japon} \\
  \TitleAndYears{Stage d'ing\'{e}nieur informaticien}{06 2008 - 09 2008} \\
  Stage d'ing\'{e}nieur \`{a} l'I.R.V.S. (laboratory For Intelligent Robots \& Vision System) \\
  \begin{itemize}
  	\item Reconstruction 3D d'une sc\`{e}ne sur GPU (GLSL) utilisant un algorithme
  		de st\'{e}r\'{e}ovision et 4 cam\'{e}ras monochromes.
  	\item D\'{e}veloppement d'une application de visualisation (C++, Qt).
  \end{itemize}
}

%%%%%%%%%%%%%%%%%%%%%%%%%%%%%%%%%%%%%%%%%%%%%%%%%%%%%%%%%%
%                Education
%%%%%%%%%%%%%%%%%%%%%%%%%%%%%%%%%%%%%%%%%%%%%%%%%%%%%%%%%%
\IfLanguageName{english} {
  \section{{\faGraduationCap} Education}
} {
  \section{{\faGraduationCap} \'{E}ducation}
}

\PlaceAndLocation{ENSEIRB-MATMECA}{Bordeaux, France} \\
\IfLanguageName{english} {
  \TitleAndYears{B.S./M.S. in Computer Science}{09 2006 - 10 2009} \\
  \TitleAndYears{Specialty in HPC (High Performance Computing)}{} \\
} {
  \TitleAndYears{Dipl\^{o}me d'Ing\'{e}nieur en Informatique}{09 2006 - 10 2009} \\
  \TitleAndYears{Option PRCD (Parall\'{e}lisme, R\'{e}gulation et Calcul Distribu\'{e})}{} \\
}

%%%%%%%%%%%%%%%%%%%%%%%%%%%%%%%%%%%%%%%%%%%%%%%%%%%%%%%%%%
%                Skills
%%%%%%%%%%%%%%%%%%%%%%%%%%%%%%%%%%%%%%%%%%%%%%%%%%%%%%%%%%
\IfLanguageName{english} {
  \section{{\faCheck} Skills}
} {
  \section{{\faCheck} Comp\'{e}tences}
}

\IfLanguageName{english} {
  \begin{itemize}
    \item {\em Competence}: Programming, CI, Documentation, Tooling, Image Pipeline, Drivers, Robotics \\
    \item {\em Programming Languages}: Bash, Kernel C, C++, Python, Java, .Net, Docker \\
    \item {\em Programming Libraries}: STL, OpenCV, Boost, V4L2, ffmpeg \\
    \item {\em Languages}: French (native), English (fluent), Japanese (beginner), Spanish (beginner) \\
  \end{itemize}
} {
  \begin{itemize}
    \item {\em Comp\'{e}tences}: Programmation, CI, Documentation, Outillage, Drivers, Vision, Robotique \\
    \item {\em Langages de Programmation}: Kernel C, C++, Bash, Python, Java, .Net, Docker \\
    \item {\em Biblioth\`{e}ques Logicielles}: STL, Qt, OpenCV, Boost, V4L2, ffmpeg \\
    \item {\em Langues}: Fran\c{c}ais (natif), Anglais (fluent), Espagnol (d\'{e}butant), Japonais (d\'{e}butant) \\
  \end{itemize}
}

%%%%%%%%%%%%%%%%%%%%%%%%%%%%%%%%%%%%%%%%%%%%%%%%%%%%%%%%%%
%                References
%%%%%%%%%%%%%%%%%%%%%%%%%%%%%%%%%%%%%%%%%%%%%%%%%%%%%%%%%%
\IfLanguageName{english} {
  \section{{\faAddressBook} References}
} {
  \section{{\faAddressBook} R\'{e}f\'{e}rences}
}

\begin{itemize}
\item \faGraduationCap~ \textbf{Dr. RABAUD Vincent}\\
\faBriefcase~ Google LLC \\
\faEnvelope~ \href{mailto:vrabaud@google.com}{\texttt{vrabaud@google.com}}\\

\item \faGraduationCap~ \textbf{Dr. PERRON Laurent}\\
\faBriefcase~ Google LLC \\
\faEnvelope~ \href{mailto:lperron@google.com}{\texttt{lperron@google.com}}\\
\end{itemize}

\end{document}
